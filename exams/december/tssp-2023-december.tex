% arara: xelatex
\documentclass[12pt]{article} % размер шрифта

\usepackage{tikz} % картинки в tikz
\usepackage{microtype} % свешивание пунктуации

\usepackage{array} % для столбцов фиксированной ширины

\usepackage{url} % для вставки ссылок \url{...}

\usepackage{indentfirst} % отступ в первом параграфе

\usepackage{sectsty} % для центрирования названий частей
\allsectionsfont{\centering} % приказываем центрировать все sections

\usepackage{amsthm} % теоремы и доказательства

\theoremstyle{definition} % прямой шрифт в условии теорем
\newtheorem{theorem}{Теорема}[section]


\usepackage{amsmath, amssymb} % куча стандартных математических плюшек

\usepackage[top=2cm, left=1.5cm, right=1.5cm, bottom=2cm]{geometry} % размер текста на странице

\usepackage{lastpage} % чтобы узнать номер последней страницы

\usepackage{enumitem} % дополнительные плюшки для списков
%  например \begin{enumerate}[resume] позволяет продолжить нумерацию в новом списке
\usepackage{caption} % подписи к картинкам без плавающего окружения figure


\usepackage{fancyhdr} % весёлые колонтитулы
\pagestyle{fancy}
\lhead{Panda, Stochastic Processes}
\chead{}
\rhead{2023-12-22}
\lfoot{}
\cfoot{DO NOT PANIC}
\rfoot{}
\renewcommand{\headrulewidth}{0.4pt}
\renewcommand{\footrulewidth}{0.4pt}



\usepackage{todonotes} % для вставки в документ заметок о том, что осталось сделать
% \todo{Здесь надо коэффициенты исправить}
% \missingfigure{Здесь будет картина Последний день Помпеи}
% команда \listoftodos — печатает все поставленные \todo'шки

\usepackage{booktabs} % красивые таблицы
% заповеди из документации:
% 1. Не используйте вертикальные линии
% 2. Не используйте двойные линии
% 3. Единицы измерения помещайте в шапку таблицы
% 4. Не сокращайте .1 вместо 0.1
% 5. Повторяющееся значение повторяйте, а не говорите "то же"

\usepackage{fontspec} % поддержка разных шрифтов
\usepackage{polyglossia} % поддержка разных языков

\setmainlanguage{english}
\setotherlanguages{russian}

\setmainfont{Linux Libertine O} % выбираем шрифт
% если Linux Libertine не установлен, то
% можно также попробовать Helvetica, Arial, Cambria и т.Д.

% чтобы использовать шрифт Linux Libertine на личном компе,
% его надо предварительно скачать по ссылке
% http://www.linuxlibertine.org/index.php?id=91&L=1

% на сервисах типа sharelatex.com этот шрифт есть :)

\newfontfamily{\cyrillicfonttt}{Linux Libertine O}
% пояснение зачем нужно шаманство с \newfontfamily
% http://tex.stackexchange.com/questions/91507/

\AddEnumerateCounter{\asbuk}{\russian@alph}{щ} % для списков с русскими буквами
%\setlist[enumerate, 2]{label=\asbuk\cdot),ref=\asbuk\cdot} % списки уровня 2 будут буквами а) б) ...

%% эконометрические и вероятностные сокращения
\DeclareMathOperator{\Cov}{Cov}
\DeclareMathOperator{\Corr}{Corr}
\DeclareMathOperator{\Var}{Var}
\DeclareMathOperator{\E}{\mathbb{E}}
\DeclareMathOperator{\grad}{grad}
\newcommand \cN{\mathcal{N}}
\newcommand \RR{\mathbb{R}}
\let\P\relax
\DeclareMathOperator{\P}{\mathbb{P}}

\newcommand \putyourname{\fbox{
    \begin{minipage}{42em}
      Name, group no:\vspace*{3ex}\par
      \noindent\dotfill\vspace{2mm}
    \end{minipage}
  }
}




\begin{document}


\putyourname

\begin{enumerate}
    \item {[10 points]} Let $(X_t)$ be independent identically distributed random variables 
    with $\P(X_i = -1) = 0.4$ and $\P(X_i = +1) = 0.6$. Consider the sum $S_t = X_1 + X_2 + \ldots + X_t$.
 \begin{enumerate}
    \item {[3]} Is $S_t$ a martingale?
    \item {[7]} Find all constants $c$ such that $M_t = c^{S_t}$ is a martingale. 
 \end{enumerate}

\end{enumerate}

 \newpage
 \putyourname

\begin{enumerate}[resume]
\item {[10 points]} Let $a(t)$ be a deterministic function, $M_t = a(t) \cos (3W_t)$ and $(W_t)$ is a Wiener process.

\begin{enumerate}
    \item {[4]} Find $dM_t$. 
    \item {[6]} Find a non-zero function $a(t)$ such that $(M_t)$ is a martingale.
\end{enumerate}
 
\end{enumerate}

\newpage
\putyourname

\begin{enumerate}[resume]
\item {[10 points]} You have two correlated Wiener processes, $(A_t)$ and $(B_t)$, with $\Corr(A_t - A_s, B_t - B_s) = \rho$ for all $t > s$.

Split the time interval $[0;t]$ into $n$ small segments of equal length. 
Let $\Delta^A_i$ be the increment of the Wiener process $(A_t)$ on the $i$-th small segment, i.e. $\Delta^A_i = A(it/n) - A((i-1)t/n)$.
Let $\Delta^B_i$ be the increment of the Wiener process $(B_t)$ on the $i$-th small segment.

Consider the sum of cross-products, $S_n = \sum_{i=1}^n \Delta^A_i \Delta^B_i$.
\begin{enumerate}
    \item {[3]} Find $\E(S_n)$.
    \item {[4]} Does $\Var(S_n)$ tend to $0$ when $n \to \infty$?
    \item {[2]} Find the mean square limit of $S_n$. 
    \item {[1]} How would you write this limit in a short hand notation with $dA_t$ and $dB_t$?
\end{enumerate}

\end{enumerate}
 
\newpage
\putyourname

\begin{enumerate}[resume]
\item {[10 points]} The process $(X_t)$ has $X_0 = 2024$, $dX_t = W_t^2 dW_t + W_t dt$, where $(W_t)$ is a Wiener process.
\begin{enumerate}
    \item {[2]} Is $(X_t)$ a martingale?
    \item {[4]} Find $d(X_t W_t)$.
    \item {[4]} Find $\Cov(X_t, W_t)$.
\end{enumerate}

\end{enumerate}
    
 
\newpage
\putyourname

\begin{enumerate}[resume]
\item {[10 points]} Consider two-period binomial model with initial share price $S_0 = 600$. 
Up and down share price multipliers are $u=1.2$, $d=0.9$, risk-free interest rate is $r = 0.05$ per period. 

The option pays you the maximal share price $X_2 = \max\{S_0, S_1, S_2\}$ at $t=2$.

\begin{enumerate}
    \item {[4]} Find the risk neutral probabilities. 
    \item {[6]} Find the current price $X_0$ of this option. 
%    \item How many shares should I have at $t=1$ in the «down» state of the world to replicate the option?
\end{enumerate}
\end{enumerate}
 
\newpage
\putyourname

\begin{enumerate}[resume]
\item {[10 points]}  Consider the framework of Black and Scholes model with riskless rate $r$, volatility $\sigma$ and initial share price $S_0$. 

Find the current price $X_0$ of an option that pays you $X_T = 1$ at fixed time $T$ if $S_T \geq 2 S_0$.

Hint: you may use the standard normal cumulative distribution function in your answer.


\end{enumerate}



\end{document}
