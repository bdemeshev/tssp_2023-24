% arara: xelatex
\documentclass[12pt]{article}

\usepackage{physics}


\usepackage{tikz} % картинки в tikz
\usepackage{microtype} % свешивание пунктуации

\usepackage{array} % для столбцов фиксированной ширины

\usepackage{indentfirst} % отступ в первом параграфе

\usepackage{sectsty} % для центрирования названий частей
\allsectionsfont{\centering}

\usepackage{amsmath, amsfonts, amssymb} % куча стандартных математических плюшек

\usepackage{comment}

\usepackage[top=2cm, left=1.2cm, right=1.2cm, bottom=2cm]{geometry} % размер текста на странице

\usepackage{lastpage} % чтобы узнать номер последней страницы

\usepackage{enumitem} % дополнительные плюшки для списков
%  например \begin{enumerate}[resume] позволяет продолжить нумерацию в новом списке
\usepackage{caption}

\usepackage{url} % to use \url{link to web}

\usepackage{fancyhdr} % весёлые колонтитулы
\pagestyle{fancy}
\lhead{Time Series and Stochastic Processes}
\chead{}
\rhead{Midterm, 2023-11-27-retake-v2}
\lfoot{}
\cfoot{}
\rfoot{}
\renewcommand{\headrulewidth}{0.4pt}
\renewcommand{\footrulewidth}{0.4pt}



\usepackage{todonotes} % для вставки в документ заметок о том, что осталось сделать
% \todo{Здесь надо коэффициенты исправить}
% \missingfigure{Здесь будет Последний день Помпеи}
% \listoftodos - печатает все поставленные \todo'шки


% более красивые таблицы
\usepackage{booktabs}
% заповеди из докупентации:
% 1. Не используйте вертикальные линни
% 2. Не используйте двойные линии
% 3. Единицы измерения - в шапку таблицы
% 4. Не сокращайте .1 вместо 0.1
% 5. Повторяющееся значение повторяйте, а не говорите "то же"



\usepackage{fontspec}
\usepackage{polyglossia}

\setmainlanguage{english}
\setotherlanguages{english}

% download "Linux Libertine" fonts:
% http://www.linuxlibertine.org/index.php?id=91&L=1
\setmainfont{Linux Libertine O} % or Helvetica, Arial, Cambria
% why do we need \newfontfamily:
% http://tex.stackexchange.com/questions/91507/
\newfontfamily{\cyrillicfonttt}{Linux Libertine O}

%\AddEnumerateCounter{\asbuk}{\russian@alph}{щ} % для списков с русскими буквами
%\setlist[enumerate, 2]{label=\asbuk*),ref=\asbuk*}

%% эконометрические сокращения
\DeclareMathOperator{\Cov}{\mathbb{C}ov}
\DeclareMathOperator{\Corr}{\mathbb{C}orr}
\DeclareMathOperator{\Var}{\mathbb{V}ar}

\let\P\relax
\DeclareMathOperator{\P}{\mathbb{P}}
\DeclareMathOperator{\plim}{\mathrm{plim}}

\DeclareMathOperator{\E}{\mathbb{E}}
% \DeclareMathOperator{\tr}{trace}
\DeclareMathOperator{\card}{card}
\DeclareMathOperator{\pCorr}{\mathrm{p}\mathbb{C}\mathrm{orr}}


\newcommand \hb{\hat{\beta}}
\newcommand \hs{\hat{\sigma}}
\newcommand \htheta{\hat{\theta}}
\newcommand \s{\sigma}
\newcommand \hy{\hat{y}}
\newcommand \hY{\hat{Y}}
\newcommand \e{\varepsilon}
\newcommand \he{\hat{\e}}
\newcommand \z{z}
\newcommand \hVar{\widehat{\Var}}
\newcommand \hCorr{\widehat{\Corr}}
\newcommand \hCov{\widehat{\Cov}}
\newcommand \cN{\mathcal{N}}
\newcommand \RR{\mathbb{R}}
\newcommand \NN{\mathbb{N}}
\newcommand{\cF}{\mathcal{F}}
\newcommand{\cH}{\mathcal{H}}


\begin{document}

\begin{enumerate}
  \item The hedgehog Melissa starts at the vertex $A$ of a triangle $\Delta ABC$.
  Each minute she randomly moves to an adjacent vertex with probabilities $\P(A \to B) = 0.7$, 
  $\P(A \to C) = 0.3$, $\P(B \to C) = \P(B \to A) = 0.5$,  $\P(C \to B) = 0.5$,  $\P(C \to A) = 0$.

  \begin{enumerate}
    \item What is the probability that she will be in vertex $C$ after 3 steps?
    \item Write down the transition matrix of this Markov chain. 
    \item What is the expected time to get from the state $A$ back to it?
  \end{enumerate}
  
  \item The number of players $N$ who will win the lottery
  is a random variable with probability mass function $\P(N = k + 1) = \exp(-1)/k!$ for $k\geq 0$.
  Each player will get a random prize $X_i \sim U[0;2]$.
  All random variables are independent. 
  Let $S$ be the sum of all the prizes. 

  \begin{enumerate}
    \item Find $\Var(S \mid N)$ and conditional moment generating function $M_{S\mid N}(u)$ for fixed value of $N$.
    \item Find the unconditional moment generating function $M_S(u)$.
  \end{enumerate}

  Note: you don't need to calculate the value in (c). 
  
  \item Consider the stochastic process $(X_n)$, where $X_0$ is uniform on $[0;1]$ and
  $X_n$ is uniformly selected on $[0; X_{n-1}]$ given $X_{n-1}$.

  \begin{enumerate}
    \item Find $\E(X_n)$ and $\Var(X_n)$.
    \item Find the probability limit $\plim X_n$.
  \end{enumerate}
  
  \item Students arrive in the Grusha café according to the Poisson arrival process $(X_t)$ 
  with constant rate $\lambda$. Let's measure time in minutes.

  The probability of no visitors during $5$ minutes is $0.10$. 
  
  \begin{enumerate}
    \item Find the value of $\lambda$.
    \item Plot the probability $\P(X_{2t} = X_{t})$ as a function of $t$.
    \item Plot the variance $\Var(X_{2t} - X_{t})$ as a function of $t$.
  \end{enumerate}

  \item 
  The random variables $X_1$ and $X_2$ are independent and normally distributed, 
  $X_1 \sim \cN(1;1)$, $X_2 \sim \cN(2;2)$. 
  I choose $X_1$ with probability $0.4$ and $X_2$ with probability $0.6$ without knowing their values.
  
  Casino pays me the value $Y$ that is equal to the chosen random variable. 

  Let the indicator $I$ be equal to $1$ if I choose $X_1$ and $0$ otherwise. 

  \begin{enumerate}
    \item Express $Y$ in terms of $X_1$, $X_2$ and $I$.
    \item Find $\E(I +Y \mid Y)$ and $\Var(I +Y \mid Y)$.
    \item Find $\Cov(I, Y)$.
  \end{enumerate}

  \item The joint distribution of $X$ and $Y$ is given in the table
    
    \begin{tabular}{*{4}{c}}
    \toprule
    & $X=-1$ & $X=0$ & $X=1$ \\
    \midrule
    $Y=-1$ & 0.1 & 0.2 & 0.3  \\
    $Y=1$ & 0.2 & 0.1 & 0.1  \\
    \bottomrule
    \end{tabular}
    
    \begin{enumerate}
     \item Explicitely find the $\sigma$-algebra $\sigma(\max\{X+Y, 0\})$.
     \item How many elements are there in $\sigma(X - Y)$?
    \end{enumerate}
    
\end{enumerate}

\end{document}

