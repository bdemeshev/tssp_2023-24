% arara: xelatex
\documentclass[12pt]{article}

\usepackage{physics}


\usepackage{tikz} % картинки в tikz
\usepackage{microtype} % свешивание пунктуации

\usepackage{array} % для столбцов фиксированной ширины

\usepackage{indentfirst} % отступ в первом параграфе

\usepackage{sectsty} % для центрирования названий частей
\allsectionsfont{\centering}

\usepackage{amsmath, amsfonts, amssymb} % куча стандартных математических плюшек

\usepackage{comment}

\usepackage[top=2cm, left=1.2cm, right=1.2cm, bottom=2cm]{geometry} % размер текста на странице

\usepackage{lastpage} % чтобы узнать номер последней страницы

\usepackage{enumitem} % дополнительные плюшки для списков
%  например \begin{enumerate}[resume] позволяет продолжить нумерацию в новом списке
\usepackage{caption}

\usepackage{url} % to use \url{link to web}

\usepackage{fancyhdr} % весёлые колонтитулы
\pagestyle{fancy}
\lhead{Time Series and Stochastic Processes}
\chead{}
\rhead{March Midterm, 2024-03-05}
\lfoot{}
\cfoot{}
\rfoot{}
\renewcommand{\headrulewidth}{0.4pt}
\renewcommand{\footrulewidth}{0.4pt}



\usepackage{todonotes} % для вставки в документ заметок о том, что осталось сделать
% \todo{Здесь надо коэффициенты исправить}
% \missingfigure{Здесь будет Последний день Помпеи}
% \listoftodos - печатает все поставленные \todo'шки


% более красивые таблицы
\usepackage{booktabs}
% заповеди из докупентации:
% 1. Не используйте вертикальные линни
% 2. Не используйте двойные линии
% 3. Единицы измерения - в шапку таблицы
% 4. Не сокращайте .1 вместо 0.1
% 5. Повторяющееся значение повторяйте, а не говорите "то же"



\usepackage{fontspec}
\usepackage{polyglossia}

\setmainlanguage{english}
\setotherlanguages{english}

% download "Linux Libertine" fonts:
% http://www.linuxlibertine.org/index.php?id=91&L=1
\setmainfont{Linux Libertine O} % or Helvetica, Arial, Cambria
% why do we need \newfontfamily:
% http://tex.stackexchange.com/questions/91507/
\newfontfamily{\cyrillicfonttt}{Linux Libertine O}

%\AddEnumerateCounter{\asbuk}{\russian@alph}{щ} % для списков с русскими буквами
%\setlist[enumerate, 2]{label=\asbuk*),ref=\asbuk*}

%% эконометрические сокращения
\DeclareMathOperator{\Cov}{\mathbb{C}ov}
\DeclareMathOperator{\Corr}{\mathbb{C}orr}
\DeclareMathOperator{\Var}{\mathbb{V}ar}

\let\P\relax
\DeclareMathOperator{\P}{\mathbb{P}}
\DeclareMathOperator{\plim}{\mathrm{plim}}

\DeclareMathOperator{\E}{\mathbb{E}}
% \DeclareMathOperator{\tr}{trace}
\DeclareMathOperator{\card}{card}
\DeclareMathOperator{\pCorr}{\mathrm{p}\mathbb{C}\mathrm{orr}}


\newcommand \hb{\hat{\beta}}
\newcommand \hs{\hat{\sigma}}
\newcommand \htheta{\hat{\theta}}
\newcommand \s{\sigma}
\newcommand \hy{\hat{y}}
\newcommand \hY{\hat{Y}}
\newcommand \e{\varepsilon}
\newcommand \he{\hat{\e}}
\newcommand \z{z}
\newcommand \hVar{\widehat{\Var}}
\newcommand \hCorr{\widehat{\Corr}}
\newcommand \hCov{\widehat{\Cov}}
\newcommand \cN{\mathcal{N}}
\newcommand \RR{\mathbb{R}}
\newcommand \NN{\mathbb{N}}
\newcommand{\cF}{\mathcal{F}}
\newcommand{\cH}{\mathcal{H}}


\newcommand \putyourname{\fbox{
    \begin{minipage}{42em}
      Name, group no:\vspace*{3ex}\par
      \noindent\dotfill\vspace{2mm}
    \end{minipage}
  }
}



\begin{document}

\putyourname

\begin{enumerate}
\item Consider $MA(2)$ process given by 
\[
y_t = 5 + u_t + 2u_{t-1} + 4 u_{t-2},
\]
where $(u_t)$ is a white noise with $\Var(u_t) = \sigma^2$.

\begin{enumerate}
  \item {[1]} Find the expected value $\E(y_t)$.
  \item {[7]} Find the autocorrelation function $\rho_k = \Corr(y_t, y_{t-k})$.
  \item {[2]} Is the process $(y_t)$ stationary?
\end{enumerate}

\end{enumerate}

\newpage
\putyourname

\begin{enumerate}[resume]
  \item Consider $MA(2)$ process given by 
\[
y_t = 5 + u_t + 2u_{t-1} + 4 u_{t-2},
\]
where $u_t$ are normal independent random variables with $\Var(u_t) = 4$.

You know that $u_{100} = 2$ and $u_{99} = -1$.

\begin{enumerate}
  \item {[5]} Find the 95\% predictive interval for $y_{101}$.
  \item {[5]} Find the 95\% predictive interval for $y_{1000001}$.
\end{enumerate}

  
\end{enumerate}


\newpage
\putyourname

\begin{enumerate}[resume]
  \item The stationary process $(y_t)$ has autocorrelation function $\rho_k = 0.2^k$ and expected value $100$.
\begin{enumerate}
  \item {[7]} Find the first two values of the partial autocorrelation function, $\phi_{11}$ and $\phi_{22}$.
  \item {[3]} Provide a possible linear recurrence equation for this process. 
  Your equation may include $y_t$, its lags and a white noise process $(u_t)$.
\end{enumerate}
  
\end{enumerate}
  


\newpage
\putyourname

\begin{enumerate}[resume]
  \item Consider the equation $y_t = 5 + 2.5 y_{t-1} - y_{t-2} + u_t$, where $(u_t)$ is a white noise process. 
  \begin{enumerate}
    \item {[3]} Find the roots of the corresponding characteristic equation. 
    \item {[4]} Rewrite the process as $A(L)(y_t - \mu) = u_t$. 
    You should explicitely write the lag polynomial $A(L)$ and the value of $\mu$.
    \item {[1]} How many non-stationary solutions does the equation have?
    \item {[1]} How many stationary solutions does the equation have?
    \item {[1]} How many stationary solutions of the $MA(\infty)$ form with respect to $(u_t)$ does the equation have?
  \end{enumerate}
  
\end{enumerate}

\newpage
\putyourname

\begin{enumerate}[resume]
  \item {[10]} The semi-annual $(y_t)$ is modelled by $ETS(ANA)$ process:
    
  \[
  \begin{cases}
      u_t \sim \cN(0; 4) \\
      s_t = s_{t-2} + 0.1 u_t \\
      \ell_t = \ell_{t-1} + 0.3 u_t \\
      y_t = \ell_{t-1} + s_{t-2} + u_t \\
  \end{cases}    
  \]
  
  Given that $s_{100} = 3$, $s_{99} = -2$, $\ell_{100} = 100$ find 95\% predictive interval for $y_{102}$. 
  
  
\end{enumerate}

\newpage
\putyourname

\begin{enumerate}[resume]
  \item {[10]} The semi-annual $(y_t)$ is modelled by $ETS(ANA)$ process:
    
  \[
  \begin{cases}
      u_t \sim \cN(0; 4) \\
      s_t = s_{t-2} + 0.1 u_t \\
      \ell_t = \ell_{t-1} + 0.3 u_t \\
      y_t = \ell_{t-1} + s_{t-2} + u_t \\
      \ell_0 = 100, s_0 = -3, s_{-1} = 3 \\
  \end{cases}    
  \]

  Check whether the process $(y_t)$ is stationary. 

  
\end{enumerate}



\end{document}

